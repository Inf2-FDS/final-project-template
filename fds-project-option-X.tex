\documentclass[11pt,a4paper]{article}
\usepackage[utf8]{inputenc}
\usepackage[T1]{fontenc}
%\usepackage{gentium}
\usepackage{mathptmx} % Use Times Font

\usepackage{graphicx} % Required for including pictures
\usepackage{hyperref} % Format links for pdf
\usepackage[british]{babel}
\usepackage{biblatex}
\usepackage[autostyle=true]{csquotes}
\MakeInnerQuote{"}
\addbibresource{references.bib}
\usepackage{booktabs} % Used so that tables generated by pandas
                      % to_latex() work correctly

\frenchspacing % No double spacing between sentences
\usepackage[margin=1in]{geometry}

\usepackage[all]{nowidow} % Tries to remove widows
\usepackage[protrusion=true,expansion=true]{microtype} % Improves typography, load after fontpackage is selected

\usepackage{lipsum} % Used for inserting dummy 'Lorem ipsum' text into the template
\usepackage{pdfpages}

\title{PUT THE TITLE OF YOUR PROJECT HERE}
\author{PUT YOUR EXAM NUMBER HERE}

\begin{document}

\maketitle

%% INSTRUCTIONS:
%%
%% 1. Create your own copy of this Overleaf project. You can either edit your report
%% using:
%%
%%    a. Overleaf professional, a collaborative LaTeX editor. You can click
%%       "Copy Project" from the Overleaf menu to create a version where you have
%%       read and write permissions. See the following for documentation:
%%       https://www.overleaf.com/edu/edinburgh and
%%       https://uoe.sharepoint.com/:f:/r/sites/digitalskillsandtraining/Shared%20Documents/LaTeX/LaTeX%20for%20Beginners%20using%20Overleaf?csf=1&web=1&e=cPqTI3
%%
%%    b. A LaTeX editor on a PC. For this option, you can download the source
%%       of this project as a zip (via the Overleaf menu).
%% 
%% 2. Keep the section and paragraph headings as they
%%    are. Keeping the headings makes the report a lot
%%    easier for the markers to read, and making things easy for
%%    markers is always beneficial.
%%
%% 3. Commentted text give more guidance about what to put in each section
%%
%% 4. In various sections, we have included examples of how to cite
%%    previous work and include figures and tables. You should delete or
%%    comment out this text. 
%%
%% 5. The word limit for the Overview section is mandatory. For the
%%    other sections word limits are suggested.
%%
%% 6. The page limits of 6 pages of the main text must be strictly adhered to.
%%    References (but not figures) can go on page 7.
%%
%% 7. Make sure to include the code file at the end of the document -
%%    instructions on how to do so are in the project spec and at the
%%    end of the document.

\section{Overview}
% 250 words maximum

\section{Introduction}
% Suggested 400 words

\paragraph{Context and motivation}

% What is the area of this data science study, and why is it
% interesting to investigate?

\paragraph{Previous work}

% Brief description of any previous work in this area (e.g., in the
% media, or scientific literature or blogs).

"Hello"

E.g.~Recent surveys show that most students prefer final projects to
final exams \cite{Space2021}.

\paragraph{Objectives}

% What questions are you setting out to answer.

\section{Data}
% Suggested 300 words

\paragraph{Data provenance} 
% Who created the dataset(s)? How you have obtained it (e.g., file or
% web scraping), and do the Terms and Conditions allow you to use
% obtain the data for the project?

\paragraph{Data description} 
% Data description, e.g. variables in each table, number of records.

\paragraph{Data processing}
% Description of how you have processed the data, e.g., cleaning,
% removing missing values, joining tables.

\section{Exploration and analysis}
% Suggested 500 words

% Provide a data science analysis of the paper, including:

% - At least one visualisation and other visualisations or tables
%   required to illustrate your findings
%
%   - Description of how you have analysed the data to address the
%   question(s) posed in the Introduction. This analysis may use the
%   statistical and ML methods learned in FDS, but it is not required
%   to use inferential statistics or ML if they do not help address
%   the question.
%
% - Interpretation of the findings


% EXAMPLE OF HOW TO INCLUDE FIGURES
% 't' means "try to position at the top of the page"
\begin{figure}[t]
  \centering
  % This iamge was created 6in wide, so that's the size we include it
  % at
  \includegraphics[width=6in]{example1-large.png}
  \caption{Demonstration figure. This caption explains more about the
    figure. Note that the font size of the labels in the plot is 8pt,
    which is obtained by the settings as shown in the Jupyter
    notebook.}
  \label{fds-project-template:fig:example1}
\end{figure}

% EXAMPLE OF HOW TO INCLUDE TABLES
% 'b' means "try to position at the bottom of the page"
\begin{table}[b]
  \centering
  \caption{Excerpt from Scottish Index of Multiple Deprivation, 2016 edition.
    \url{https://simd.scot}. You may put more information in the
    caption.}
  % NOTE: THIS TABLE WAS GENERATED FROM PANDAS
  \begin{tabular}{llrl}
\toprule
Data Zone & Council area & Total population & Income rate \\
\midrule
S01006506 & Aberdeen City & 904 & 7\% \\
S01006507 & Aberdeen City & 830 & 7\% \\
S01006508 & Aberdeen City & 694 & 5\% \\
S01006509 & Aberdeen City & 573 & 5\% \\
S01006510 & Aberdeen City & 676 & 10\% \\
\bottomrule
\end{tabular}

  \label{tab:example1}
\end{table}

\textbf{Example of how to include figures and tables:}  Use the \texttt{figure} and \texttt{table} environments and refer to figures and tables in text, for example:

Figure~\ref{fds-project-template:fig:example1} shows some data.
Figure~\ref{fds-project-template:fig:example2} shows the same data,
but more compactly. You might find it helpful to have compact figures
if you are getting close to the 6-page limit.

The Scottish Index of Multiple Deprivation contains information on various statistics relating to deprivation in small areas of Scotland (Table~\ref{tab:example1}).

\textbf{Example of how to use equations:} You can use equations like
this:
\begin{equation}
  \label{fds-project-template:eq:1}
  \overline{x} = \sum_{i=1}^n x_i
\end{equation}
or maths inline: $E=mc^2$. However, you do not need to reexplain techniques that you have learned in the course -- assume the reader understands linear regression, logistic regression K-nearest neighbours etc. Remember to explain any symbols use, e.g.~``$n$ is the number of data points and $x_i$ is the value of the $i$th data point.''.

% 't' means "try to position here"
\begin{figure}[h]
  \centering
  % This iamge was created 3in wide, so that's the size we include it
  % at
  \includegraphics[width=3in]{example1-small.png}
  \caption{Scottish Index of Multiple Deprivation data. Note that the
    font size of the labels in the plot is 8pt, which is obtained by
    the settings as shown in the Jupyter notebook. Also note that the
    font size is the same as
    Figure~\ref{fds-project-template:fig:example1}, because we've
    included the files at the correct size.}
  \label{fds-project-template:fig:example2}
\end{figure}


\section{Discussion and conclusions}
% Suggested 400 words.

\paragraph{Summary of findings}

\paragraph{Evaluation of own work: strengths and limitations}

\paragraph{Comparison with any other related work}

E.g. ``Anscombe has also demonstrated that many patterns of data can
have the same correlation coefficient'' \cite{anscombe1973graphs}.

If you find information on Wikipedia, you can cite it. However, it is better to cite the original reference for a particular claim found on Wikipedia, which you can find in the list of references on the Wikipedia page.

The golden rule is always to cite information that has come from other
sources, to avoid plagiarism \cite{wiki:plagarism}.

\paragraph{Improvements and extensions}


\printbibliography

% DO NOT REMOVE THIS LINE
\clearpage

% Save Your Jupyter Notebook Pdf Output As FDS-Project-Code.pdf and
% put it in the same directory as this file

% DO NOT EDIT THIS LINE
\includepdf[pages=1-]{FDS-Project-Code.pdf}

\end{document}

% LocalWords:  lrrrrrrr ment Macduff Kemnay Ruchill FDS mc th fds pdf
% LocalWords:  Anscombe